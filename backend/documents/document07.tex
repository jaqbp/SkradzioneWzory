\documentclass{article}
\usepackage[utf8]{inputenc}

\title{Obszerny Dowód Matematyczny z Wzorami}
\author{Autor G}
\date{Maj 2024}

\usepackage{amsmath, amssymb, amsthm}

\theoremstyle{plain}
\newtheorem{theorem}{Twierdzenie}

\begin{document}

\maketitle

\section{Wprowadzenie}

W tym artykule udowodnimy pewne twierdzenie z analizy matematycznej.

\section{Twierdzenie}

\begin{theorem}
Dla każdej liczby rzeczywistej $x$, funkcja $f(x) = \sin^2(x) + \cos^2(x)$ przyjmuje wartość $1$.
\end{theorem}

\begin{proof}
Rozważmy funkcję trygonometryczną podniesioną do kwadratu:
\[ f(x) = \sin^2(x) + \cos^2(x) \]

Z wykorzystaniem tożsamości trygonometrycznych mamy:
\[ \sin^2(x) + \cos^2(x) = 1 \]

Tożsamość ta jest znana jako tożsamość trygonometryczna Pitagorasa. Oznacza ona, że suma kwadratów funkcji sinus i cosinus dowolnego kąta prostokątnego zawsze wynosi $1$.

Zatem dla każdej liczby rzeczywistej $x$, funkcja $f(x)$ przyjmuje wartość $1$. Co kończy dowód.
\end{proof}

\end{document}
