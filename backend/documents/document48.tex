\documentclass{article}
\usepackage{amsmath, amssymb, mathtools}

\begin{document}

\section*{Kolejne 10 Skomplikowanych Wzorów Matematycznych}

\subsection*{Wzór 121: Równanie Dirichaletta dla funkcji Laplace'a}

Równanie Dirichaletta dla funkcji Laplace'a \(u\) na obszarze \(\Omega\) to:
\[ \Delta u = f \text{ w } \Omega, \quad u = g \text{ na } \partial \Omega \]

\subsection*{Wzór 122: Funkcja Legendre'a \(P_n(x)\)}

Funkcje Legendre'a \(P_n(x)\) są rozwiązaniem równania różniczkowego:
\[ (1-x^2) \frac{d^2P_n}{dx^2} - 2x \frac{dP_n}{dx} + n(n+1)P_n = 0 \]

\subsection*{Wzór 123: Równanie Boltzmanna dla gazów rzeczywistych}

Równanie Boltzmanna dla gazów rzeczywistych opisuje zachowanie cząstek gazu i jest dane przez:
\[ \frac{\partial f}{\partial t} + \mathbf{v} \cdot \nabla_x f = Q(f, f) \]

\subsection*{Wzór 124: Nierówność Szasz-Mirakyan}

Nierówność Szasz-Mirakyan z teorii analizy funkcji określonych na zbiorze Carlesona mówi o aproksymacji funkcji przez średnie kwadratowe szeregów Fouriera.

\subsection*{Wzór 125: Równanie Eulera-Lagrange'a dla mechaniki nieba}

Równanie Eulera-Lagrange'a dla mechaniki nieba opisuje ruch ciał niebieskich i jest dane przez:
\[ \frac{d}{dt}\left(\frac{\partial L}{\partial \dot{q}_i}\right) - \frac{\partial L}{\partial q_i} = 0 \]

\subsection*{Wzór 126: Funkcja Riemanna-Siegela \(Z(t)\)}

Funkcję Riemanna-Siegela \(Z(t)\) można zdefiniować jako:
\[ Z(t) = e^{i\theta(t)}\zeta\left(\frac{1}{2} + it\right) \]

\subsection*{Wzór 127: Równanie Hamiltona dla oscylatora kwantowego}

Równanie Hamiltona dla oscylatora kwantowego opisuje stan kwantowy oscylatora i jest dane przez:
\[ \hat{H}\psi = -\frac{\hbar^2}{2m}\frac{d^2\psi}{dx^2} + \frac{1}{2}m\omega^2x^2\psi \]

\subsection*{Wzór 128: Całka Barnesa \(G(x)\)}

Całka Barnesa \(G(x)\) to całka zdefiniowana przez:
\[ G(x) = (2\pi)^{\frac{(x-1)^2}{2}}e^{-x(x-1)\gamma} \prod_{n=1}^{\infty}\left(1+\frac{x}{n}\right)^ne^{-x/n} \]

\subsection*{Wzór 129: Nierówność Poincaré dla przestrzeni Banacha}

Nierówność Poincaré dla przestrzeni Banacha mówi, że dla funkcji \(u\) spełniającej pewne warunki, istnieje stała \(C\) taka, że dla każdego obszaru \(\Omega\):
\[ \int_{\Omega} |u - u_{\Omega}|^2 \leq C \int_{\Omega} |\nabla u|^2 \]

\subsection*{Wzór 130: Twierdzenie Okaia dla równań różniczkowych cząstkowych}

Twierdzenie Okaia dla równań różniczkowych cząstkowych mówi o istnieniu rozwiązania pewnego rodzaju równań z pewnymi warunkami początkowymi.

\end{document}
