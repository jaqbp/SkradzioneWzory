\documentclass{article}
\usepackage{amsmath, amssymb, mathtools}

\begin{document}

\section*{Kolejne 10 Skomplikowanych Wzorów Matematycznych}

\subsection*{Wzór 81: Twierdzenie o niezmienniczości formy różniczkowej}

Twierdzenie o niezmienniczości formy różniczkowej mówi, że dla różniczkowalnej funkcji gładkiej \(f\) i formy różniczkowej \(\omega\) zachodzi:
\[ f^*\omega = \omega \]

\subsection*{Wzór 82: Równanie Black-Scholesa dla opcji finansowych}

Równanie Black-Scholesa opisuje zmiany w czasie wartości opcji finansowej i jest dane przez:
\[ \frac{\partial V}{\partial t} + \frac{1}{2}\sigma^2 S^2 \frac{\partial^2 V}{\partial S^2} + rS \frac{\partial V}{\partial S} - rV = 0 \]

\subsection*{Wzór 83: Funkcja Charlwood}

Funkcję Charlwood \(C(x)\) można zdefiniować jako całkę:
\[ C(x) = \int_{0}^{x} t^2\cos(t) \, dt \]

\subsection*{Wzór 84: Twierdzenie Stokesa}

Dla dowolnej \(n\)-wymiarowej rozmaitości z brzegiem, twierdzenie Stokesa mówi, że dla formy różniczkowej \(\omega\):
\[ \int_{M} d\omega = \int_{\partial M} \omega \]

\subsection*{Wzór 85: Równanie Fokkera-Plancka}

Równanie Fokkera-Plancka opisuje ewolucję dystrybucji prawdopodobieństwa cząstek w czasie i jest dane przez:
\[ \frac{\partial P}{\partial t} = -\nabla \cdot (\mathbf{F}P) + \nabla \cdot (D\nabla P) \]

\subsection*{Wzór 86: Wielomiany Laguerre'a}

Wielomiany Laguerre'a \(L_n(x)\) są rozwiązaniem równania różniczkowego:
\[ x\frac{d^2L_n}{dx^2} + (1-x)\frac{dL_n}{dx} + nL_n = 0 \]

\subsection*{Wzór 87: Równanie Riccati}

Równanie Riccati dla funkcji \(y(x)\) jest postaci:
\[ y'(x) = a(x) + b(x)y(x) + c(x)y^2(x) \]

\subsection*{Wzór 88: Macierz Pauliego}

Macierze Pauliego \(\sigma_1, \sigma_2, \sigma_3\) są używane w teorii kwantowych pól i są dane przez:
\[ \sigma_1 = \begin{bmatrix} 0 & 1 \\ 1 & 0 \end{bmatrix}, \quad
   \sigma_2 = \begin{bmatrix} 0 & -i \\ i & 0 \end{bmatrix}, \quad
   \sigma_3 = \begin{bmatrix} 1 & 0 \\ 0 & -1 \end{bmatrix} \]

\subsection*{Wzór 89: Równanie Eikonalne}

Równanie Eikonalne opisuje propagację fali i jest postaci:
\[ |\nabla u|^2 = n^2(x) \]

\subsection*{Wzór 90: Wielomiany Hermite'a}

Wielomiany Hermite'a \(H_n(x)\) są rozwiązaniem równania różniczkowego:
\[ e^{-x^2} \frac{d^2H_n}{dx^2} - 2xe^{-x^2} \frac{dH_n}{dx} + 2ne^{-x^2}H_n = 0 \]

\end{document}
