\documentclass{article}
\usepackage[utf8]{inputenc}

\title{Dowód Matematyczny 5}
\author{Autor K}
\date{Wrzesień 2024}

\usepackage{amsmath, amssymb, amsthm}

\theoremstyle{plain}
\newtheorem{theorem}{Twierdzenie}

\begin{document}

\maketitle

\section{Wprowadzenie}

W tym artykule przedstawimy dowód wzoru sumy kwadratów liczb naturalnych.

\section{Twierdzenie}

\begin{theorem}
Dla każdej liczby całkowitej dodatniej $n$, suma kwadratów liczb naturalnych $\sum_{k=1}^{n} k^2$ wynosi $\frac{n(n+1)(2n+1)}{6}$.
\end{theorem}

\begin{proof}
Rozważmy szereg kwadratowy:
\[ S_n = 1^2 + 2^2 + 3^2 + \ldots + n^2 \]

Możemy skorzystać z wzoru na sumę kwadratów liczb naturalnych, który wynosi:
\[ \sum_{k=1}^{n} k^2 = \frac{n(n+1)(2n+1)}{6} \]

Stąd, dla dowolnej liczby całkowitej dodatniej $n$, suma kwadratów liczb naturalnych $\sum_{k=1}^{n} k^2$ wynosi $\frac{n(n+1)(2n+1)}{6}$.
\end{proof}

\end{document}
