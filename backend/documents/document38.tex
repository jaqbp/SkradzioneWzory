\documentclass{article}
\usepackage{amsmath, amssymb}

\begin{document}

\section*{Kolejne 10 Różnych Wzorów Matematycznych}

\subsection*{Wzór 21: Równanie kwadratowe}

Równanie kwadratowe \(ax^2 + bx + c = 0\) ma rozwiązania:
\[ x = \frac{-b \pm \sqrt{b^2-4ac}}{2a} \]

\subsection*{Wzór 22: Szereg geometryczny}

Suma nieskończonego szeregu geometrycznego o pierwszym wyrazie \(a\) i ilorazie \(r\), dla \(|r| < 1\), to:
\[ S = \frac{a}{1-r} \]

\subsection*{Wzór 23: Własności logarytmów}

Dla każdych liczb dodatnich \(a\), \(b\) i \(c\) zachodzą następujące równości:
\begin{align*}
\log_a(b \cdot c) &= \log_a(b) + \log_a(c) \\
\log_a\left(\frac{b}{c}\right) &= \log_a(b) - \log_a(c) \\
\log_a(b^n) &= n \cdot \log_a(b)
\end{align*}

\subsection*{Wzór 24: Nierówność Bernoulliego}

Dla każdej liczby rzeczywistej \(x > -1\) i każdej liczby rzeczywistej \(n \geq 1\) zachodzi nierówność:
\[ (1 + x)^n \geq 1 + nx \]

\subsection*{Wzór 25: Twierdzenie Fermata}

Dla liczb całkowitych dodatnich \(a\), \(b\), i \(c\) równanie \(a^n + b^n = c^n\) nie ma rozwiązań, gdy \(n > 2\).

\subsection*{Wzór 26: Wzór Taylora dla funkcji e^x}

Dla każdego \(x\) zachodzi wzór Taylora funkcji \(e^x\):
\[ e^x = 1 + x + \frac{x^2}{2!} + \frac{x^3}{3!} + \frac{x^4}{4!} + \ldots \]

\subsection*{Wzór 27: Wzór Simpsona}

Dla danej funkcji \(f(x)\) i przedziału \([a, b]\), przybliżona wartość całki \(\int_{a}^{b} f(x) \,dx\) za pomocą wzoru Simpsona to:
\[ \frac{b-a}{6} \left[f(a) + 4f\left(\frac{a+b}{2}\right) + f(b)\right] \]

\subsection*{Wzór 28: Równanie falowe}

Równanie falowe dla fali dźwiękowej o prędkości \(v\) i częstości \(f\) to:
\[ v = \lambda \cdot f \]
gdzie \(\lambda\) to długość fali.

\subsection*{Wzór 29: Prawo Gaussa}

Dla dowolnego zamkniętego obszaru \(V\) w przestrzeni trójwymiarowej, fluks wektora pola grawitacyjnego \(\mathbf{g}\) przez powierzchnię \(S\) otaczającą ten obszar to:
\[ \oint_S \mathbf{g} \cdot d\mathbf{S} = -4\pi G \int_V \rho \,dV \]
gdzie \(\rho\) to gęstość masy, \(G\) to stała grawitacyjna.

\subsection*{Wzór 30: Szereg Fourier'a}

Szereg Fourier'a funkcji okresowej \(f(x)\) o okresie \(2\pi\) można zapisać jako:
\[ f(x) = a_0 + \sum_{n=1}^{\infty} \left[a_n \cos(nx) + b_n \sin(nx)\right] \]

\end{document}
