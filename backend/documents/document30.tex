\documentclass{article}
\usepackage{amsmath}

\begin{document}

\section*{Wzory z analizy matematycznej}

\subsection*{Granice}

\[
    \lim_{x \to \infty} \left(1 + \frac{1}{x}\right)^x = e
\]
\[
    \lim_{h \to 0} \frac{\sin(h)}{h} = 1
\]
\[
    \lim_{n \to \infty} \left(1 + \frac{x}{n}\right)^n = e^x
\]

\subsection*{Szereg funkcyjny}

\[
    f(x) = \sum_{n=0}^{\infty} \frac{f^{(n)}(a)}{n!}(x-a)^n
\]

\subsection*{Pochodne cząstkowe}

\[
    \frac{\partial f}{\partial x} = \lim_{h \to 0} \frac{f(x+h, y) - f(x, y)}{h}
\]
\[
    \frac{\partial^2 f}{\partial x^2} = \frac{\partial}{\partial x}\left(\frac{\partial f}{\partial x}\right)
\]

\subsection*{Równania różniczkowe}

\[
    \frac{dy}{dx} + P(x)y = Q(x)
\]
\[
    \frac{\partial^2 u}{\partial t^2} = c^2 \frac{\partial^2 u}{\partial x^2}
\]

\end{document}
