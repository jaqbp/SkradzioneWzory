\documentclass{article}
\usepackage[utf8]{inputenc}

\title{Przykładowy Dokument 5}
\author{Autor E}
\date{Marzec 2024}

\begin{document}

\maketitle

\section{Wprowadzenie}

W tym artykule prezentujemy różne podejścia do problemów teorii grafów.

\section{Definicje}

Poniżej znajdują się definicje i twierdzenia związane z teorią grafów:

1. Graf pełny:
\[ K_n = (V, E), \quad |V| = n, \quad |E| = \binom{n}{2} \]

2. Kolorowanie grafu:
\[ \chi(G) = \min \{k : \text{istnieje k-kolorowanie wierzchołków}\} \]

3. Twierdzenie Kuratowskiego:
\[ \text{Graf jest planarny} \iff \text{Nie zawiera podgrafu izomorficznego ani $K_5$, ani $K_{3,3}$.} \]

\end{document}
