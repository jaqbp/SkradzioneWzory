\documentclass{article}
\usepackage[utf8]{inputenc}

\title{Dokument Wzorcowy 3}
\author{Autor C}
\date{Styczeń 2024}

\begin{document}

\maketitle

\section{Wprowadzenie}

W tym artykule przedstawiamy różne pojęcia z dziedziny fizyki teoretycznej.

\section{Równania}

Poniżej znajdziesz kilka równań charakteryzujących się zastosowaniami w fizyce:

1. Równanie Schrödingera:
\[ i\hbar \frac{\partial}{\partial t} \Psi(\mathbf{r}, t) = -\frac{\hbar^2}{2m} \nabla^2 \Psi(\mathbf{r}, t) + V(\mathbf{r}, t) \Psi(\mathbf{r}, t) \]

2. Druga zasada dynamiki Newtona:
\[ F = m \cdot a \]

3. Równanie falowe Maxwella:
\[ \nabla \times \mathbf{E} = -\frac{\partial \mathbf{B}}{\partial t} \]

\end{document}
