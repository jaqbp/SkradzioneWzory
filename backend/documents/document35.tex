\documentclass{article}
\usepackage[utf8]{inputenc}
\usepackage{polski}
\usepackage[mathscr]{eucal}
\usepackage{amsmath}
\usepackage{amssymb}
\usepackage{MnSymbol}
\usepackage{amsfonts}




\begin{document}
\newcommand{\imp}{\Rightarrow}
\newcommand{\lub}{\vee}
\newcommand{\roz}{\setminus}
\newcommand{\zbp}{\emptyset}
\newcommand{\zbpot}{\mathcal{P}}
\newcommand{\troj}{\bigtriangleup}

\maketitle
Nad zadaniami myślałem sam.
\section*{Zadanie 1.}

\subsection*{$(\zbpot ( X \cap Y)) \cup (X \roz Y)$}
Zacznijmy od prostych rachunków na zbiorach $X$ i $Y$

\[X\cap Y = \{0,1\}\]
\[X\roz Y = \{2\}\]

Wobec tego:

\[\zbpot ( X \cap Y) = \{\zbp, \{0\}, \{1\}, \{0,1\}\} \]

\[(\zbpot ( X \cap Y)) \cup (X \roz Y) = \{\zbp, \{0\}, \{1\}, \{0,1\}\} \cup \{2\} \]

Czyli:

\[(\zbpot ( X \cap Y)) \cup (X \roz Y) = \{\zbp, 2, \{0\}, \{1\}, \{0,1\}\}  \]

\subsection*{$\zbpot (\zbpot(X\roz Y)$}

\[X\roz Y = \{2\}\]

\[\zbpot (X\roz Y) = \{\zbp,\{2\}\}\]


\section*{Zadanie 2.}

Niech:

\[A = \{0\}\]
\[B = \{1\}\]

Wtedy:

\[\zbpot (A) = \{\zbp,\{0\}\}\]
\[\zbpot (B) = \{\zbp,\{1\}\}\]

Więc:
\[ \zbpot (A) \troj \zbpot (B) = \{\{0\},\{1\}\} \]

Z drugiej strony:
\[A \troj B = \{0,1\}\]
\[\zbpot (A\troj B) = \{\zbp,\{0\},\{1\},\{0,1\}\}\]

Wobec tego równość $ \zbpot (A) \troj \zbpot (B) = \zbpot (A\troj B) $ jest fałszywa, ponieważ zbiory te mają różne elementy. $\blacksquare$


\section*{Zadanie 3.}

\subsection*{Dowód $\bigcup \zbpot(A) \subseteq A$:}

Weźmy $x$ takie, że $x \in \bigcup \zbpot(A)$. Ponieważ $\bigcup \zbpot(A)$ to suma podzbiorów $A$, to $x$ musi być elementem jednego z tych podzbiorów. Wobec tego $\exists(B \subseteq A) \ x\in B$.
\\
$B \subseteq A$, oznacza, że $\forall(y\in B)\ y\in A$. Jest to prawdą w szczególności dla $x$, co oznacza, że $x\in A$, czyli  $\bigcup \zbpot(A) \subseteq A$

\subsection*{Dowód $\bigcup \zbpot(A) \supseteq A$:}

Weźmy $x \in A$. Zauważmy, że $x$ jest w szczególności podzbiorem $A$. Wobec tego $x$ należy do sumy podzbiorów $A$, czyli $x \in \bigcup \zbpot(A)$

Oznacza to, że $A \subseteq \bigcup \zbpot(A)$
\\\\
Z zawierania $\bigcup \zbpot(A) \subseteq A$ i $\bigcup \zbpot(A) \supseteq A$ wynika, że $\bigcup \zbpot(A) = A$\\
$\blacksquare$
\end{document}
