\documentclass{article}
\usepackage{amsmath, amssymb}

\begin{document}

\section*{10 Różnych Dowodów Matematycznych}

\subsection*{Dowód 11: Twierdzenie Pitagorasa}

Dla każdego trójkąta prostokątnego o przyprostokątnych \(a\) i \(b\) oraz przeciwprostokątnej \(c\), zachodzi równość:
\[ a^2 + b^2 = c^2 \]

\subsection*{Dowód 12: Suma kątów w trójkącie}

Suma miar kątów w dowolnym trójkącie wynosi \(180^\circ\).

\subsection*{Dowód 13: Nierówność trójkąta}

Dla dowolnych trzech boków \(a\), \(b\) i \(c\) trójkąta zachodzi nierówność:
\[ a + b > c \]

\subsection*{Dowód 14: Wzór Herona}

Pole trójkąta o bokach \(a\), \(b\) i \(c\) można obliczyć za pomocą wzoru Herona:
\[ \text{Pole} = \sqrt{s(s-a)(s-b)(s-c)} \]
gdzie \(s\) to połowa obwodu trójkąta, czyli \(s = \frac{a+b+c}{2}\).

\subsection*{Dowód 15: Wzór na sumę szeregów arytmetycznych}

Suma \(S_n\) pierwszych \(n\) wyrazów szeregu arytmetycznego o pierwszym wyrazie \(a_1\) i różnicy \(d\) to:
\[ S_n = \frac{n}{2}[2a_1 + (n-1)d] \]

\subsection*{Dowód 16: Suma szeregów geometrycznych}

Suma nieskończonego szeregu geometrycznego o pierwszym wyrazie \(a\) i ilorazie \(r\), dla \(|r| < 1\), to:
\[ S = \frac{a}{1-r} \]

\subsection*{Dowód 17: Twierdzenie o wartości średniej (Cauchy'ego)}

Dla dowolnych funkcji ciągłych \(f(x)\) i \(g(x)\) na przedziale \([a, b]\), gdzie \(g(x) \neq 0\) dla \(x\) z tego przedziału, istnieje \(c\) z przedziału \((a, b)\) taki, że:
\[ \frac{f(c)}{g(c)} = \frac{f(b) - f(a)}{g(b) - g(a)} \]

\subsection*{Dowód 18: Wzór na liczbę pi \(\pi\)}

Wzór Leibniza na liczbę \(\pi\):
\[ \pi = 4 \left(1 - \frac{1}{3} + \frac{1}{5} - \frac{1}{7} + \frac{1}{9} - \ldots \right) \]

\subsection*{Dowód 19: Wzór na sumę szeregu harmonicznego}

Szereg harmoniczny \(H_n\) danego wzoru:
\[ H_n = 1 + \frac{1}{2} + \frac{1}{3} + \ldots + \frac{1}{n} \]
ma ograniczenie:
\[ H_n < 1 + \frac{1}{2} + \frac{1}{2} + \ldots + \frac{1}{2^{k-1}} = 2 \]

\subsection*{Dowód 20: Twierdzenie o funkcji odwrotnej}

Jeśli funkcja \(f\) jest różnowartościowa, ciągła i rosnąca na przedziale \([a, b]\), to jej funkcja odwrotna \(f^{-1}\) istnieje i również jest różnowartościowa, ciągła oraz rosnąca na przedziale \([f(a), f(b)]\).

\end{document}
