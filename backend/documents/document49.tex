\documentclass{article}
\usepackage{amsmath, amssymb, mathtools}

\begin{document}

\section*{Kolejne 10 Skomplikowanych Wzorów Matematycznych}

\subsection*{Wzór 131: Równanie Yang-Millsa}

Równanie Yang-Millsa w teorii pola opisuje skomplikowane oddziaływania między polami gauge'owymi i jest dane przez:
\[ D_\mu F^{\mu\nu} = J^\nu \]

\subsection*{Wzór 132: Nierówność Cheegera}

Nierówność Cheegera w teorii grafów mówi o rozmiarze najmniejszego przekroju w grafie ważonym, co jest związane z własnościami geometrycznymi grafu.

\subsection*{Wzór 133: Równanie Kadomtseva-Petviasha dla superprzewodników}

Równanie Kadomtseva-Petviasha opisuje skojarzenie nadprzewodnictwa z magnetyzmem i jest dane przez:
\[ \Delta(x)\left(\frac{\hbar^2}{2m}(-i\nabla-\frac{e\mathbf{A}}{c})^2 + V(x) - \mu\right) + \alpha|\Delta(x)|^2\Delta(x) = 0 \]

\subsection*{Wzór 134: Funkcja Belyi}

Funkcję Belyi można zdefiniować jako funkcję meromorficzną na powierzchni Riemanna, której jedynymi biegunami są \(\infty\) i trzy dowolne inne liczby zespolone.

\subsection*{Wzór 135: Równanie Bernoulliego dla analitycznych funkcji}

Równanie Bernoulliego dla analitycznych funkcji \(f(z)\) mówi, że jeśli \(f(z)\) jest analityczna na dysku otwartym zawierającym punkt \(z_0\), to:
\[ f(z) = \sum_{n=0}^{\infty} c_n(z-z_0)^n \]

\subsection*{Wzór 136: Twierdzenie Weierstrassa o faktoryzacji całkowitej}

Twierdzenie Weierstrassa o faktoryzacji całkowitej mówi, że każda funkcja całkowita można przedstawić jako iloczyn funkcji elementarnych i czynnika eksponencjalnego.

\subsection*{Wzór 137: Nierówność Morse'a}

Nierówność Morse'a w topologii różniczkowej mówi o krytycznych punktach funkcji Morse'a i ich wpływie na strukturę topologiczną rozmaitości.

\subsection*{Wzór 138: Równanie Hamiltona dla układu dynamicznego}

Równanie Hamiltona dla układu dynamicznego opisuje ewolucję układu i jest dane przez:
\[ \dot{q}_i = \frac{\partial H}{\partial p_i}, \quad \dot{p}_i = -\frac{\partial H}{\partial q_i} \]

\subsection*{Wzór 139: Wielomiany Hermite'a-Appell'a}

Wielomiany Hermite'a-Appell'a są rozszerzeniem wielomianów Hermite'a i są używane w kwantowej mechanice statystycznej.

\subsection*{Wzór 140: Równanie Minkowskiego dla czasoprzestrzeni}

Równanie Minkowskiego opisuje czasoprzestrzeń w teorii względności i jest dane przez:
\[ ds^2 = -c^2dt^2 + dx^2 + dy^2 + dz^2 \]

\end{document}
