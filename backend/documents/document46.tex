\documentclass{article}
\usepackage{amsmath, amssymb, mathtools}

\begin{document}

\section*{Kolejne 10 Skomplikowanych Wzorów Matematycznych}

\subsection*{Wzór 91: Całka Fresnela dla fazy zespolonej}

Całka Fresnela dla fazy zespolonej \(z\) to:
\[ C(z) = \int_{0}^{z} e^{i\pi t^2} \, dt \]

\subsection*{Wzór 92: Równanie Kortewega-de Vriesa (KdV)}

Równanie Kortewega-de Vriesa opisuje rozchodzenie się fal w jednowymiarowych ośrodkach i jest dane przez:
\[ u_t + 6uu_x + u_{xxx} = 0 \]

\subsection*{Wzór 93: Funkcja Weierstrassa ℘}

Funkcję Weierstrassa ℘ można zdefiniować za pomocą równania różniczkowego:
\[ (\wp')^2 = 4(\wp - g_2)(\wp - g_3)(\wp - g_1) \]

\subsection*{Wzór 94: Twierdzenie Atiyaha-Singera}

Twierdzenie Atiyaha-Singera z teorii operatorów liniowych mówi, że indeks pewnego operatora jest równy śladowi pewnego operatora.

\subsection*{Wzór 95: Wzór Funka-Hecke'a dla funkcji modularnych}

Wzór Funka-Hecke'a dla funkcji modularnych \(f\) to:
\[ f(\gamma \cdot \tau) = (c\tau + d)^{-k} \sum_{\substack{m \in \mathbb{Z}\\(m, c)=1}} f\left(\frac{a\tau + b}{c\tau + d}\right) \]

\subsection*{Wzór 96: Funkcja Riemanna zeta \(\zeta(s)\)}

Funkcję Riemanna zeta \(\zeta(s)\) można zdefiniować jako szereg:
\[ \zeta(s) = 1^s + 2^{-s} + 3^{-s} + \ldots \]

\subsection*{Wzór 97: Równanie Diraca dla funkcji falowej}

Równanie Diraca opisuje dynamikę funkcji falowej elektronu i jest dane przez:
\[ i\hbar\frac{\partial \psi}{\partial t} = -i\hbar c\boldsymbol{\alpha}\cdot\nabla\psi + \beta mc^2\psi \]

\subsection*{Wzór 98: Wzór Landauera-Büttikera dla przewodnictwa elektrycznego}

Wzór Landauera-Büttikera opisuje przewodnictwo elektryczne w nanostrukturach i jest dany przez:
\[ G = \frac{e^2}{h}T(E) \sum_{n,m} \left[f(E_n) - f(E_m)\right] \]

\subsection*{Wzór 99: Wielomiany Chebysheva}

Wielomiany Chebysheva \(T_n(x)\) są rozwiązaniem równania różniczkowego:
\[ (1-x^2) y'' - xy' + n^2y = 0 \]

\subsection*{Wzór 100: Twierdzenie de Finettiego}

Twierdzenie de Finettiego z teorii prawdopodobieństwa mówi, że dowolny nieskończony wymierny rozkład prawdopodobieństwa można przedstawić jako średnią ważoną dystrybuant rozkładów skończonych.

\end{document}
