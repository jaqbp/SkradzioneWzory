\documentclass{article}
\usepackage{amsmath, amssymb, mathtools}

\begin{document}

\section*{Kolejne 10 Skomplikowanych Wzorów Matematycznych}

\subsection*{Wzór 111: Równanie Naviera-Stokesa dla nieściśliwego płynu}

Równanie Naviera-Stokesa dla nieściśliwego płynu opisuje ruch płynu i jest dane przez:
\[ \frac{\partial \mathbf{u}}{\partial t} + (\mathbf{u} \cdot \nabla) \mathbf{u} = -\frac{1}{\rho} \nabla p + \nu \nabla^2 \mathbf{u} \]

\subsection*{Wzór 112: Szereg Fouriera dla funkcji okresowej}

Szereg Fouriera dla funkcji okresowej \(f(x)\) to:
\[ f(x) = a_0 + \sum_{n=1}^{\infty} \left(a_n \cos\left(\frac{2\pi n x}{T}\right) + b_n \sin\left(\frac{2\pi n x}{T}\right)\right) \]

\subsection*{Wzór 113: Funkcja Möbiusa \(\mu(n)\)}

Funkcję Möbiusa \(\mu(n)\) można zdefiniować rekurencyjnie:
\[ \mu(n) = 
\begin{cases} 
1 & \text{jeśli } n = 1 \\
(-1)^k & \text{jeśli } n \text{ ma } k \text{ różnych czynników pierwszych} \\
0 & \text{jeśli } n \text{ ma kwadratowy czynnik pierwszy}
\end{cases}
\]

\subsection*{Wzór 114: Równanie dyfuzji z reakcją chemiczną}

Równanie dyfuzji z reakcją chemiczną to:
\[ \frac{\partial c}{\partial t} = D \nabla^2 c - k c^n \]

\subsection*{Wzór 115: Równanie Kohna-Shama dla teorii funkcjonału gęstości}

Równanie Kohna-Shama w teorii funkcjonału gęstości opisuje problem kwantomechaniczny wielu elektronów i jest dane przez:
\[ -\frac{\hbar^2}{2m} \nabla^2 \psi_i + V_{\text{eff}}[\rho] \psi_i = \epsilon_i \psi_i \]

\subsection*{Wzór 116: Twierdzenie Gleasona}

Twierdzenie Gleasona mówi, że każda dodatnia, nieujemnie jednorodna funkcja na przestrzeni operacyjnej przypisanej kwantowemu systemowi skończenie wymiarowemu może być przedstawiona jako ślad operatora gęstości.

\subsection*{Wzór 117: Całka Kurzweila-Henstocka}

Całka Kurzweila-Henstocka jest bardziej ogólnym uogólnieniem całki Riemanna i jest definiowana za pomocą funkcji dystrybucji.

\subsection*{Wzór 118: Równanie Lotki-Volterry dla układu drapieżnik-ofiara}

Równanie Lotki-Volterry opisuje interakcję między populacją drapieżników (\(y\)) a populacją ofiar (\(x\)) i jest dane przez:
\[ \frac{dx}{dt} = \alpha x - \beta xy, \quad \frac{dy}{dt} = \delta xy - \gamma y \]

\subsection*{Wzór 119: Nierówność Isoperimetryczna}

Nierówność isoperimetryczna dla krzywych na płaszczyźnie mówi, że krzywa o danej długości, która otacza największy obszar, jest okręgiem.

\subsection*{Wzór 120: Twierdzenie Liouville'a dla funkcji harmonicznych}

Twierdzenie Liouville'a dla funkcji harmonicznych mówi, że każda ograniczona funkcja harmoniczna na całej przestrzeni \(\mathbb{R}^n\) musi być funkcją stałą.

\end{document}
