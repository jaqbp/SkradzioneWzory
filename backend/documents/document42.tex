\documentclass{article}
\usepackage{amsmath, amssymb}

\begin{document}

\section*{Kolejne 10 Różnych Wzorów Matematycznych}

\subsection*{Wzór 61: Tożsamość trygonometryczna cotangensa sumy kątów}

Dla dowolnych kątów \(\alpha\) i \(\beta\), tożsamość trygonometryczna cotangensa sumy kątów to:
\[ \cot(\alpha + \beta) = \frac{\cot\alpha \cdot \cot\beta - 1}{\cot\alpha + \cot\beta} \]

\subsection*{Wzór 62: Równanie różniczkowe Bernoulliego}

Równanie różniczkowe Bernoulliego to równanie postaci:
\[ y' + P(x)y = Q(x)y^n \]

\subsection*{Wzór 63: Własności pierwiastków trzeciego stopnia}

Dla dowolnej liczby rzeczywistej \(x\), pierwiastki trzeciego stopnia liczby \(x^3\) to:
\[ \sqrt[3]{x^3} = x \]

\subsection*{Wzór 64: Twierdzenie o resztach z dzielenia przez \(n\)-tą jednostkową pierwiastek z jedności}

Dla dowolnych liczb całkowitych \(a\) i \(n\), reszta z dzielenia \(a^n\) przez \(n\)-tą jednostkową pierwiastek z jedności to:
\[ a^n \equiv a \pmod{n} \]

\subsection*{Wzór 65: Równanie Laplace'a w cylindrze}

Dla funkcji \(u(r, \theta, z)\) opisującej pole potencjału w cylindrze, równanie Laplace'a w cylindrze to:
\[ \nabla^2 u = \frac{1}{r}\frac{\partial}{\partial r}\left(r\frac{\partial u}{\partial r}\right) + \frac{1}{r^2}\frac{\partial^2 u}{\partial \theta^2} + \frac{\partial^2 u}{\partial z^2} = 0 \]

\subsection*{Wzór 66: Nierówność Jensen-Shannon}

Dla dowolnych dwóch rozkładów prawdopodobieństwa \(P\) i \(Q\), nierówność Jensen-Shannon mówi, że:
\[ D_{JS}(P\|Q) \leq \frac{1}{2}D_{KL}(P\|M) + \frac{1}{2}D_{KL}(Q\|M) \]
gdzie \(M\) to średni rozkład między \(P\) i \(Q\).

\subsection*{Wzór 67: Wzór MacLaurina dla funkcji \(\sin(x)\)}

Wzór MacLaurina dla funkcji trygonometrycznej \(\sin(x)\) to:
\[ \sin(x) = x - \frac{x^3}{3!} + \frac{x^5}{5!} - \frac{x^7}{7!} + \ldots \]

\subsection*{Wzór 68: Równanie dyfuzji-reakcji}

Dla funkcji gęstości \(u(x, t)\) opisującej dyfuzję-reakcję, równanie dyfuzji-reakcji w jednym wymiarze to:
\[ \frac{\partial u}{\partial t} = D \frac{\partial^2 u}{\partial x^2} + R(u) \]

\subsection*{Wzór 69: Równanie Eulera dla funkcji eliptycznych}

Dla funkcji eliptycznych \(F(\phi, k)\) i \(K(k)\), równanie Eulera to:
\[ F(\phi, k) = \frac{\pi}{2} \sum_{n=0}^{\infty} \left(\frac{\sin((2n+1)\phi)}{2n+1}\right) \left(\frac{2n+1}{\sqrt{1-k^2}\sin\phi}\right)^{2n+1} \]

\subsection*{Wzór 70: Tożsamość Weierstrassa}

Tożsamość Weierstrassa dla funkcji \(x^2\) to:
\[ x^2 = \frac{\pi^2}{3} + 4\sum_{n=1}^{\infty} (-1)^n \frac{\cos(2n\pi x)}{(2n)^2} \]

\end{document}
