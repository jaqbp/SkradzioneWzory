\documentclass{article}
\begin{document}
Wzór 148:
\[ \frac{d}{dx}\left(\sec(x)\right) = \sec(x) \tan(x) \]

Wzór 149:
\[ \frac{\partial}{\partial x}\left(\int_{a}^{x} f(t) \,dt\right) = f(x) \]

Wzór 150:
\[ \oint_C \mathbf{F} \cdot d\mathbf{r} = \int_{a}^{b} \mathbf{F}(\mathbf{r}(t)) \cdot \mathbf{r}'(t) \,dt \]

Dowód 4: \\
Rozważmy ciąg arytmetyczny o pierwszym wyrazie \( a \), różnicy \( d \) i \( n \) wyrazach. Suma \( S \) tego ciągu to
\[ S = \frac{n}{2}(2a + (n-1)d) \].

\end{document}
