\documentclass{article}
\usepackage{amsmath, amssymb, mathtools}

\begin{document}

\section*{Kolejne 10 Najbardziej Skomplikowanych Wzorów Matematycznych}

\subsection*{Wzór 141: Równanie Maxwella dla elektromagnetyzmu}

Równanie Maxwella dla elektromagnetyzmu opisuje ewolucję pól elektrycznych (\(E\)) i magnetycznych (\(B\)) w czasoprzestrzeni i jest dane przez:
\begin{align*}
  \nabla \cdot \mathbf{E} &= \frac{\rho}{\varepsilon_0} \\
  \nabla \cdot \mathbf{B} &= 0 \\
  \nabla \times \mathbf{E} &= -\frac{\partial \mathbf{B}}{\partial t} \\
  \nabla \times \mathbf{B} &= \mu_0\mathbf{J} + \mu_0\varepsilon_0\frac{\partial \mathbf{E}}{\partial t}
\end{align*}

\subsection*{Wzór 142: Nierówność Naviera-Stokesa}

Nierówność Naviera-Stokesa dla równań płynów opisuje zachowanie płynów i jest dana przez:
\[ \int_{\Omega} |\mathbf{u}|^p \, dx \leq C\left(\int_{\Omega} |\nabla \mathbf{u}|^p \, dx\right)^\alpha \left(\int_{\Omega} |\nabla^2 \mathbf{u}|^p \, dx\right)^{1-\alpha} \]

\subsection*{Wzór 143: Równanie Einstein'a dla teorii ogólnej względności}

Równanie pola Einsteina dla teorii ogólnej względności opisuje zakrzywienie czasoprzestrzeni przez masę i energię i jest dane przez:
\[ G_{\mu\nu} + \Lambda g_{\mu\nu} = \frac{8\pi G}{c^4} T_{\mu\nu} \]

\subsection*{Wzór 144: Funkcja Blocha dla teorii ciał stałych}

Funkcję Blocha w teorii ciał stałych można opisać jako:
\[ \psi_{k}(r) = e^{ik\cdot r}u_k(r) \]

\subsection*{Wzór 145: Równanie Schrödingera dla wielu cząstek}

Równanie Schrödingera dla wielu cząstek opisuje stan kwantowy układu i jest dane przez:
\[ i\hbar \frac{\partial}{\partial t} \Psi(\mathbf{r}_1, \mathbf{r}_2, \ldots, \mathbf{r}_N, t) = \hat{H}\Psi(\mathbf{r}_1, \mathbf{r}_2, \ldots, \mathbf{r}_N, t) \]

\subsection*{Wzór 146: Równanie Diraca dla spinorów}

Równanie Diraca dla spinorów opisuje dynamikę cząstek o spinie 1/2 i jest dane przez:
\[ (i\gamma^\mu \partial_\mu - m)\psi = 0 \]

\subsection*{Wzór 147: Równanie Hamiltona-Jacobiego dla mechaniki kwantowej}

Równanie Hamiltona-Jacobiego dla mechaniki kwantowej opisuje rozwinięcie funkcji falowej (\(\Psi\)) w czasie i jest dane przez:
\[ \frac{\partial S}{\partial t} + H(\nabla S, \mathbf{q}) = 0 \]

\subsection*{Wzór 148: Równanie Fokkera-Planka dla procesu stochastycznego}

Równanie Fokkera-Planka opisuje rozkład prawdopodobieństwa dla procesu stochastycznego i jest dane przez:
\[ \frac{\partial P}{\partial t} = -\nabla \cdot (\mathbf{F}P) + \frac{1}{2}\nabla \cdot (\nabla P) \]

\subsection*{Wzór 149: Równanie Vlasova dla plazmy}

Równanie Vlasova dla plazmy opisuje ewolucję funkcji rozkładu elektronów i jonów w czasie i przestrzeni i jest dane przez:
\[ \frac{\partial f}{\partial t} + \mathbf{v} \cdot \nabla f + \frac{e}{m}(\mathbf{E} + \mathbf{v} \times \mathbf{B}) \cdot \nabla_v f = 0 \]

\subsection*{Wzór 150: Równanie Kadomtseva-Petviasha dla nadprzewodników z dziurami}

Równanie Kadomtseva-Petviasha dla nadprzewodników z dziurami opisuje współczynniki przerwy energetycznej w zależności od temperatury i jest dane przez:
\[ \Delta(x)\left(-\frac{\hbar^2}{2m}(\nabla-\frac{e\mathbf{A}}{c})^2 + V(x) - \mu\right) + \alpha|\Delta(x)|^2\Delta(x) = 0 \]

\end{document}
