\documentclass{article}
\usepackage[utf8]{inputenc}
\usepackage{polski}
\usepackage[mathscr]{eucal}
\usepackage{amsmath}
\usepackage{amssymb}
\usepackage{MnSymbol}
\usepackage{amsfonts}



\begin{document}
\newcommand{\imp}{\Rightarrow}
\newcommand{\lub}{\vee}
\newcommand{\roz}{\setminus}
\newcommand{\zbp}{\emptyset}
\newcommand{\zbpot}{\mathcal{P}}
\newcommand{\troj}{\bigtriangleup}
\newcommand{\nat}{\mathbb{N}}
\newcommand{\rze}{\mathbb{R}}
\newcommand{\wegde}{\wedge}
\newcommand{\eps}{\varepsilon}
\maketitle
Nad zadaniami myślałem sam.
\section*{Zadanie 1.}
\subsection*{Dowód $(A \times B) \roz \{\langle x,y\rangle\} \subseteq (( A\roz  \{x\}) \times B) \cup (A\times (B\roz\{y\})$}
Weźmy takie $m,n$ że: $\langle m,n \rangle \in (A \times B) \roz \{\langle x,y\rangle\} $.\\ Wtedy: \[ m \in A \wedge n \in  B \wedge \lnot (m = x \wedge  n = y)\]
Po użyciu prawa de Morgana mamy: \[ m \in A \wedge n \in  B \wedge  (m \neq x \vee  n \neq y)\]
Korzystając z rozdzielności koniunkcji względem alternatywy otrzymujemy: 
\[ (m \in A \wedge m \neq x \wedge n \in  B  ) \vee ( m \in A \wedge n \in  B \wedge n \neq y)\]
Zamieniając na iloczyn kartezjański otrzymujemy:
\[\langle m,n \rangle \in (( A\roz  \{x\}) \times B) \cup (A\times (B\roz\{y\})\]
Co oznacza, że $(A \times B) \roz \{\langle x,y\rangle\} \subseteq (( A\roz  \{x\}) \times B) \cup (A\times (B\roz\{y\})$


\subsection*{Dowód $(A \times B) \roz \{\langle x,y\rangle\} \supseteq (( A\roz  \{x\}) \times B) \cup (A\times (B\roz\{y\})$}
Weźmy takie $m,n$ że $\langle m,n\rangle \in ((A\roz  \{x\}) \times B) \cup (A\times (B\roz\{y\})$\\
Wtedy:\[ (m \in A \wedge m \neq x \wedge n \in  B  ) \vee ( m \in A \wedge n \in  B \wedge n \neq y)\]
Używając rozdzielności koniunkcji względem alternatywy otrzymujemy:
\[ m \in A \wedge n \in  B \wedge  (m \neq x \vee  n \neq y)\]
Po zastosowaniu prawa de Morgana zostaje:
\[ m \in A \wedge n \in  B \wedge \lnot (m = x \wedge  n = y)\]
Zamieniając na iloczyn kartezjański mamy:\[\langle m,n \rangle \in (A \times B) \roz \{\langle x,y\rangle\}\]
Czyli $(A \times B) \roz \{\langle x,y\rangle\} \supseteq (( A\roz  \{x\}) \times B) \cup (A\times (B\roz\{y\})$
\subsection*{Podsumowanie}
Ponieważ mamy następujące zawierania: \[(A \times B) \roz \{\langle x,y\rangle\} \subseteq (( A\roz  \{x\}) \times B) \cup (A\times (B\roz\{y\})\] i \[(A \times B) \roz \{\langle x,y\rangle\} \supseteq (( A\roz  \{x\}) \times B) \cup (A\times (B\roz\{y\})\] prawdziwa jest równość \[(A \times B) \roz \{\langle x,y\rangle\} = (( A\roz  \{x\}) \times B) \cup (A\times (B\roz\{y\})\]
\blacksquare

\section*{Zadanie 2.}
\subsection*{(a)}
Ponieważ funkcje $f$($x$) $=$ $\frac{1}{x+1}$  i $g$ ($x$) $= 1-x$ są malejące, to największa wartość wyrażenia $1 - \frac{1}{n+1}$ jest dla najmniejszej wartości  $\frac{1}{n+1}$, czyli największej wartości $n$.\\


Wobec tego \[\forall_{n\in \nat} \ 1 - \frac{1}{n+1} \leq \lim_{n\to\infty} 1 - \frac{1}{n+1} = 1 - 0 = 1\]

Z drugiej strony, ponieważ funkcja $f$ ($x$) $= 1+x$ jest rosnąca, to najmniejsza wartość wyrażenia $1+n$ jest dla najmniejszych $n$.

Wobec tego \[\forall_{n\in \nat} \ 1+n \geq 1+0 = 1 \]

Korzystając z dwóch powyższych faktów mamy:
\[A_n = \{ x\in \rze : \ \frac{1}{n+1} \leq 1 \leq x \leq 1 \leq 1+n\} \]
Czyli:
\[\bigcap_{n=0}^{\infty}A_n \in [1,1] = 1\]

\subsection*{(b)}

Ponieważ funkcje $f$($x$) $=$ $\frac{1}{x+1}$  i $g$ ($x$) $= 1-x$ są malejące, to najmniejsza wartość wyrażenia $1 - \frac{1}{n+1}$ jest dla największej wartości  $\frac{1}{n+1}$, czyli najmniejszej $n$.\\



Wobec tego \[\forall_{n\in \nat} \ 1 - \frac{1}{n+1} \geq 1 - \frac{1}{1+1} = \frac{1}{2}\]

Z drugiej strony, ponieważ funkcja $f$ ($x$) $= 1+x$ jest rosnąca, to największa wartość wyrażenia $1+n$ jest dla największych $n$.

Wobec tego \[\forall_{n\in \nat} \ 1+n \leq 1+31 = 32 \]

Korzystając z dwóch powyższych faktów mamy:
\[A_n = \{ x\in \rze : \frac{1}{2} \leq \frac{1}{n+1} \leq x  \leq 1+n \leq 32\} \]
Czyli:
\[\bigcup_{n\in I}A_n \in [\frac{1}{2},32] \]

\section*{Zadanie 3.}
\subsection*{Dowód "$\Rightarrow$"}

Wiedząc, że $A$ jest sumą pewnej rodziny odcinków otwartych, weźmy odcinek, należący do rodziny, której sumą jest $A$. Nazwijmy go $y$. Ponieważ $y$ jest odncinkiem otwartym, to ma swój początek i koniec, które oznaczymy odpowiednio $y_a$, $y_b$.\\
Zauważmy, wobec tego, że \[\forall_{k\in \rze  \wegde k\in y} \ y_a < k < y_b\]
Niech $\eps$ = min\{$\frac{k - y_a}{2}, \frac{y_b - k}{2}$\} , jest ono oczywiście dodatnie, możemy je przypisać każdej liczbie $k$ ponieważ między dwoma liczbami rzeczywistymi jest nieskończenie wiele liczb rzeczywistych.
Wtedy: 
\[y_a < k-\eps < k < k+\eps < y_b\]
Więc \[\forall_{x\in A} \exists_{\eps > 0} (x-\eps,x+\eps) \subset A\]

\subsection*{Dowód "\Leftarrow"}
Wiedząc, że $\forall_{x\in A} \exists_{\eps > 0} (x-\eps,x+\eps) \subset A$ rozpatrzmy sumę wszystkich tych odcinków. Suma ta jest jest równa zbiorowi $A$. Wobec tego zbiór $A$ jest sumą pewnej rodziny odcinków otwartych 

\subsection*{Podsumowanie}
Z implikacji $"\Rightarrow" \ $ oraz $\ "\Leftarrow"$ wynika, że $A \subset R$ jest sumą pewnej rodziny odcinków otwartych wtedy i tylko wtedy, gdy $\forall_{x\in A} \exists_{\eps > 0} (x-\eps,x+\eps) \subset A$

\blacksquare

\end{document}
