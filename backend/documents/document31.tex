\documentclass{article}
\usepackage{amsmath}

\begin{document}

\section*{Wzory z analizy matematycznej}

\subsection*{Granice}

\[
    \lim_{x \to 0} \frac{\sin x}{x} = 1
\]
\[
    \lim_{x \to \infty} \left(1 + \frac{1}{x}\right)^x = e
\]
\[
    \lim_{n \to \infty} \left(1 + \frac{x}{n}\right)^n = e^x
\]

\subsection*{Pochodne}

\[
    \frac{d}{dx} e^x = e^x
\]
\[
    \frac{d}{dx} \ln x = \frac{1}{x}
\]
\[
    \frac{d}{dx} \tan x = \sec^2 x
\]

\subsection*{Całki}

\[
    \int e^x \,dx = e^x + C
\]
\[
    \int \frac{1}{x} \,dx = \ln |x| + C
\]
\[
    \int \sec^2 x \,dx = \tan x + C
\]

\subsection*{Równania różniczkowe}

\[
    y' + P(x)y = Q(x)
\]
\[
    \frac{\partial^2 u}{\partial t^2} = c^2 \frac{\partial^2 u}{\partial x^2}
\]

\end{document}
