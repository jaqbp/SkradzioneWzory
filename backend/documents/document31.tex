\documentclass{article}
\usepackage[utf8]{inputenc}
\usepackage{polski}
\usepackage[mathscr]{eucal}
\usepackage{amsmath}
\usepackage{amssymb}
\usepackage{MnSymbol}
\usepackage{amsfonts}
\usepackage{graphicx}





\begin{document}
\newcommand{\imp}{\Rightarrow}
\newcommand{\lub}{\vee}
\newcommand{\roz}{\setminus}
\newcommand{\zbp}{\emptyset}
\newcommand{\zbpot}{\mathcal{P}}
\newcommand{\troj}{\bigtriangleup}
\newcommand{\nat}{\mathbb{N}}
\newcommand{\calk}{\mathbb{Z}}
\newcommand{\rze}{\mathbb{R}}
\newcommand{\wegde}{\wedge}
\newcommand{\eps}{\varepsilon}
\maketitle
Nad zadaniami myślałem sam.
\section*{Zadanie 1.}
Wiedząc, że $y=x-1$ mamy: 
\[f(x,y) = \langle x + x - 1, x^2 - (x-1)^2\rangle\]
Redukując otrzymujemy:
\[f(x,y) = \langle 2x - 1, 2x - 1\rangle\]
Zauważmy, że obie współrzędne są takie same. Wobec tego $f[A]$ to zbiór punktów o dwóch tych samych współrzędnych, czyli taki, który zawiera się w prostej $f(x)=x$. Wygląda on tak:
\[\includegraphics[width=5cm]{zadanie 1.png}\]


\section*{Zadanie 2.}

$f^{-1}[B]$ oznacza, że szukamy takich $\langle x,y\rangle$, że $x+y \in [0,1]\ \wedge \ x^2 - y^2 >0 $ Wobec tego:
\[x+y \geq 0 \ \wegde \ x + y \leq 1 \ \wegde  \ (x-y)(x+y) >0 \]
Czyli:
\[ y\geq -x \ \wegde \ y \leq 1-x \ \wegde \ ((x>y \wegde x > -y)  \lub (x<-y \wedge x<y)) \]
Co po zredukowaniu daje:
\[ y\geq -x \ \wegde \ y \leq 1-x \ \wegde \ ((x>y \wegde -x < y)  \lub (-x>y \wedge x<y)) \]
Po narysowaniu w płaszczyźnie, zbiór wygląda tak:
\[\includegraphics[width=5cm]{zadanie 2.png}\]


\section*{Zadanie 3.}

\subsection*{Dowód "$\Rightarrow$"}
Ponieważ $|X| = |Y|$, to istnieje bijekcyjna funkcja $f: X \rightarrow Y$. Wobec tego:
\[\forall_{y \in Y} \exists!_{x \in X} \ f(x) = y \]
Niech $\forall_{x \in X} \ A_x = \{ x,f(x)\} $. Weźmy $\mathcal{F}: \forall_{x \in X} A_x \in \mathcal{F} $. Wtedy dla każdego $A \in \mathcal{F}$ do zbioru $A$ należy jeden element ze zbioru $X$ i jeden element ze zbioru $Y$. Ponieważ $f$ jest różnowartościowa, zbiory $A_x$ są parami rozłączne. Ponadto: 
\[\bigcup \mathcal{F} = \bigcup_{x\in X} A_x= \bigcup_{x\in X} \{x,f(x)\} =  (\bigcup_{x\in X} \{x\} \cup \bigcup_{x\in X} \{f(x)\}) \]
Ponieważ zbiory $X$,$Y$ są rozłączne, $f(x) \in Y$ oraz $f$ jest "na" (z bijekcji) to $\bigcup_{x\in X} \{f(x)\} = \bigcup_{y\in Y} \{y\}$.  Wówczas:
\[ \bigcup \mathcal{F} = (\bigcup_{x\in X} \{x\} \cup \bigcup_{x\in X} \{f(x)\})= (\bigcup_{x\in X} \{x\} \cup \bigcup_{y\in Y} \{y\}) = X\cup Y\]
Kończy to dowód implikacji "$\Rightarrow$". 
\subsection*{Dowód "$\Leftarrow$"}
\subsubsection*{Dowód $|X|\geq |Y|$}
Weźmy funkcję $f: X \rightarrow Y$. Niech $\forall_{A_x \in \mathcal{F}} \ A_x = \{x,f(x)\}$. Wiemy, że \[\bigcup \mathcal{F} = \bigcup_{x \in X} A_x = \bigcup_{x\in X} \{x,f(x)\} =  (\bigcup_{x\in X} \{x\} \cup \bigcup_{x\in X} \{f(x)\}) = X \cup Y\]
Ponieważ $X$,$Y$ są rozłączne, to: \[\bigcup_{x\in X} \{f(x)\} = Y\]
Wobec tego funkcja $f: X \rightarrow Y$ jest "na". Czyli istotnie $|X|\geq |Y|$.

\subsubsection*{Dowód $|X|\leq |Y|$}
Weźmy funkcję $g: Y \rightarrow X$. Niech $\forall_{A_y \in \mathcal{F}} \ A_y = \{g(y),y\}$. Wiemy, że \[\bigcup \mathcal{F} = \bigcup_{y \in Y} A_y = \bigcup_{y\in Y} \{g(y),y\} =  (\bigcup_{y\in Y} \{g(y)\} \cup \bigcup_{y\in Y} \{y\}) = X \cup Y\]
Ponieważ $X$,$Y$ są rozłączne, to: \[\bigcup_{y\in Y} \{g(y)\} = X\]
Wobec tego funkcja $g: Y \rightarrow X$ jest "na". Czyli istotnie $|X|\leq |Y|$. \\\
Ponieważ $|X|\geq |Y|$ i $|X|\leq |Y|$, to z twierdzenia Cantora-Bernsteina $|X| = |Y|$, co kończy dowód implikacji "$\Leftarrow$".

Z implikacji  "$\Rightarrow$" oraz "$\Leftarrow$" wynika, że  dla zbiorów rozłącznych $X$ i $Y$  $|X| = |Y |$ wtedy i tylko wtedy,
gdy istnieje taka rodzina $\mathcal{F}$ zbiorów dwuelementowych parami rozłącznych, że są spełnione następujące
dwa warunki:
\\\
(i) Dla każdego $A \in  \mathcal{F}$ do zbioru A należy jeden element ze zbioru $X$ i jeden element ze zbioru $Y$ 
\\\
(ii) 
$\bigcup \mathcal{F} = X \cup Y$

\end{document}
