\documentclass{article}
\usepackage[utf8]{inputenc}
\usepackage{polski}
\usepackage[mathscr]{eucal}
\usepackage{amsmath}
\usepackage{amssymb}
\usepackage{MnSymbol}
\usepackage{amsfonts}




\begin{document}
\newcommand{\imp}{\Rightarrow}
\newcommand{\lub}{\vee}
\newcommand{\roz}{\setminus}
\newcommand{\zbp}{\emptyset}
\newcommand{\zbpot}{\mathcal{P}}
\newcommand{\troj}{\bigtriangleup}
\newcommand{\nat}{\mathbb{N}}
\newcommand{\rze}{\mathbb{R}}
\newcommand{\wegde}{\wedge}
\newcommand{\eps}{\varepsilon}
\maketitle
Nad zadaniami myślałem sam. W rozwiązaniu założyłem, że 0 należy do zbioru liczb naturalnych.
\section*{Zadanie 1.}
Niech $\forall_{ n \in \nat }$ $A_n = \{n\}$ , $B_n = \{n+1\}$
\\ Zauważmy, że $A_n \cap B_n = \zbp$  ponieważ $n\neq n+1$
\\Wobec tego: 
\[\bigcup_{n \in \nat} ( A_n \cap B_n ) = \zbp \]

Z drugiej strony: 
\[ \bigcup_{n \in \nat } A_n = \nat  \]
\[ \bigcup_{n \in \nat } B_n = \nat \roz \{0\}  \]

Więc:
\[ \bigcup_{n \in \nat } A_n \cap \bigcup_{n \in \nat } B_n = \nat \roz \{0\}  \]

Wobec tego: 
\[ \bigcup_{n \in \nat} ( A_n \cap B_n ) \neq \bigcup_{n \in \nat } A_n \cap \bigcup_{n \in \nat } B_n   \]

\section*{Zadanie 2.}


\subsection*{$\bigcup_{n \in \nat} \bigcap_{m \in \nat} A_{n,m}$}  

Ustalmy n. Niech $B_n = \bigcap_{m \in \nat} A_{n,m}$

Można zauważyć, że $A_{n,m} = [n^2,m^2]$

Rozważmy dwa przypadki:
\subsubsection*{1. $n=0$}
\[A_{n,m} = [0,m^2]\]
Więc: \[B_n = \bigcap_{m \in \nat} A_{n,m} = \{0\}\]

\subsubsection*{2. $n>0$}
Jeśli weżmiemy $m = 0$, to $A_{n,m}= [n^2,0] = \zbp$
Wobec tego:

\[	B_n = \bigcap_{m \in \nat} A_{n,m} = \zbp	\]

Oznacza to, że:
\[	\bigcup_{n \in \nat} \bigcap_{m \in \nat} A_{n,m} = \bigcup_{n \in \nat} B_n = \{0\}	\]

\subsection*{$\bigcap_{n \in \nat} \bigcup_{m \in \nat} A_{n,m}$}

Ustalmy n. Weźmy $B_n = \bigcup_{m \in \nat} A_{n,m}$
Ponieważ $A_{n,m} = [n^2,m^2]$ 

\[	B_n = \bigcup_{m \in \nat} A_{n,m} = [n^2,\infty]\	\] 

Stąd:
\[	\bigcap_{n \in \nat} \bigcup_{m \in \nat} A_{n,m} = \bigcap_{n \in \nat} B_n = \zbp	\]



\section*{Zadanie 3.}

\subsection*{(a)}
Niech  $x \in \zbpot (\rze)$, żeby $x = f(\langle A_i : i \in \nat \rangle) = \bigcap_{i \in \nat} A_i$
wystarczy, że 

\[\forall_{i \in \nat} A_i = x\]

Zawsze możemy tak zrobić, ponieważ skoro $x \in \zbpot(\rze)$ to \\
$\{x,x,x,...,x\} \in \zbpot(\rze)^{\nat}$

Więc $f$ jest „na”. 

\subsection*{(b)}
Niech $A_i = \{0\}$
$B_i = \left\{ \begin{array}{rcl}
\{0\} & \mbox{dla}
& i = 0\\ \{0,1\} & \mbox{dla} & i>0  
\end{array}\right.$

\[f(\langle A_i : i \in \nat \rangle) = \bigcap_{i \in \nat} A_i = \{0\}\]

\[f(\langle B_i : i \in \nat \rangle) = \bigcap_{i \in \nat} B_i = \{0\}\]

Wobec tego $f$ nie jest różnowartościowa, ponieważ dla różnych argumentów przyjmuje tę samą wartość



\end{document}
