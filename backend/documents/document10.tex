\documentclass{article}
\usepackage[utf8]{inputenc}

\title{Dowód Matematyczny 3}
\author{Autor I}
\date{Lipiec 2024}

\usepackage{amsmath, amssymb, amsthm}

\theoremstyle{plain}
\newtheorem{theorem}{Twierdzenie}

\begin{document}

\maketitle

\section{Wprowadzenie}

W tym artykule przedstawimy dowód wzoru sumy szeregowej.

\section{Twierdzenie}

\begin{theorem}
Dla każdej liczby całkowitej dodatniej $n$, suma szeregowa $\sum_{k=1}^{n} k$ wynosi $\frac{n(n+1)}{2}$.
\end{theorem}

\begin{proof}
Rozważmy szereg arytmetyczny:
\[ S_n = 1 + 2 + 3 + \ldots + n \]

Możemy odwrócić kolejność elementów w drugiej połowie tego szeregu:
\[ S_n = 1 + 2 + 3 + \ldots + n \]
\[ S_n = n + (n-1) + (n-2) + \ldots + 1 \]

Teraz dodajmy obie formy szeregów:
\[ 2S_n = (n+1) + (n+1) + (n+1) + \ldots + (n+1) \]

Istnieje $n$ sumowanych wyrazów równych $(n+1)$, zatem:
\[ 2S_n = n(n+1) \]

Podzielmy przez $2$ obie strony:
\[ S_n = \frac{n(n+1)}{2} \]

Co kończy dowód.
\end{proof}

\end{document}
