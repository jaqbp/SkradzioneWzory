\documentclass{article}
\usepackage[utf8]{inputenc}

\title{Dowód Matematyczny 4}
\author{Autor J}
\date{Sierpień 2024}

\usepackage{amsmath, amssymb, amsthm}

\theoremstyle{plain}
\newtheorem{theorem}{Twierdzenie}

\begin{document}

\maketitle

\section{Wprowadzenie}

W tym artykule przedstawimy dowód nierówności Cauchy'ego-Schwarza.

\section{Twierdzenie}

\begin{theorem}
Dla dowolnych wektorów $\mathbf{u}$ i $\mathbf{v}$ w przestrzeni euklidesowej, nierówność Cauchy'ego-Schwarza wyraża się wzorem:
\[ |\langle \mathbf{u}, \mathbf{v} \rangle|^2 \leq \langle \mathbf{u}, \mathbf{u} \rangle \cdot \langle \mathbf{v}, \mathbf{v} \rangle \]
\end{theorem}

\begin{proof}
Rozważmy wektory $\mathbf{u}$ i $\mathbf{v}$ w przestrzeni euklidesowej. Możemy utworzyć funkcję:
\[ f(t) = \langle t\mathbf{u} + \mathbf{v}, t\mathbf{u} + \mathbf{v} \rangle \]

Obliczmy pochodną drugiego rzędu tej funkcji względem $t$:
\[ f''(t) = 2 \langle \mathbf{u}, \mathbf{u} \rangle \]

Ponieważ $f''(t)$ jest zawsze nieujemne, funkcja $f(t)$ ma minimum globalne. Zatem, dla dowolnej wartości $t$, zachodzi nierówność:
\[ \langle t\mathbf{u} + \mathbf{v}, t\mathbf{u} + \mathbf{v} \rangle \geq 0 \]

Rozwijając tę nierówność, otrzymujemy:
\[ t^2 \langle \mathbf{u}, \mathbf{u} \rangle + 2t \langle \mathbf{u}, \mathbf{v} \rangle + \langle \mathbf{v}, \mathbf{v} \rangle \geq 0 \]

To jest równoważne:
\[ |\langle \mathbf{u}, \mathbf{v} \rangle|^2 \leq \langle \mathbf{u}, \mathbf{u} \rangle \cdot \langle \mathbf{v}, \mathbf{v} \rangle \]

Co kończy dowód.
\end{proof}

\end{document}
