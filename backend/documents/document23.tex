\documentclass{article}
\begin{document}
Wzór 121:
\[ \lim_{{n \to \infty}} \sqrt[n]{n} = 1 \]

Wzór 122:
\[ \frac{d}{dx}\left(e^{cx}\right) = c e^{cx} \]

Wzór 123:
\[ \iint_R f(x,y) \,dA = \int_{a}^{b} \int_{c}^{d} f(x,y) \,dy\,dx \]

Wzór 124:
\[ \lim_{{x \to 0}} \frac{\cos(x) - 1}{x} = 0 \]

Wzór 125:
\[ \binom{n}{k} = \binom{n-1}{k-1} + \binom{n-1}{k} \]

Wzór 126:
\[ \int_{-\infty}^{\infty} e^{-x^2} \,dx = \sqrt{\pi} \]

Wzór 127:
\[ \sum_{k=1}^{\infty} \frac{1}{k^2} = \frac{\pi^2}{6} \]

Wzór 128:
\[ \frac{d}{dx}\left(\sec(x)\right) = \sec(x) \tan(x) \]

Wzór 129:
\[ \frac{\partial}{\partial x}\left(\int_{a}^{x} f(t) \,dt\right) = f(x) \]

Dowód 1:\\
Rozważmy równanie kwadratowe \( ax^2 + bx + c = 0 \). Jeśli \( a \neq 0 \), to jego rozwiązania to
\[ x = \frac{-b \pm \sqrt{b^2-4ac}}{2a} \].

\end{document}
