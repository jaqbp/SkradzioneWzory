\documentclass{article}
\usepackage{amsmath, amssymb}

\begin{document}

\section*{Kolejne 10 Różnych Wzorów Matematycznych}

\subsection*{Wzór 31: Nierówność Cauchy'ego-Schwarza}

Dla dowolnych wektorów \(\mathbf{a} = (a_1, a_2, \ldots, a_n)\) i \(\mathbf{b} = (b_1, b_2, \ldots, b_n)\) w przestrzeni euklidesowej, nierówność Cauchy'ego-Schwarza mówi, że:
\[ \left(\sum_{i=1}^{n} a_i b_i\right)^2 \leq \left(\sum_{i=1}^{n} a_i^2\right) \left(\sum_{i=1}^{n} b_i^2\right) \]

\subsection*{Wzór 32: Wzór Stirlinga}

Dla dużych \(n\), przybliżony wzór Stirlinga dla silnii \(n!\) to:
\[ n! \approx \sqrt{2\pi n} \left(\frac{n}{e}\right)^n \]

\subsection*{Wzór 33: Własności macierzy transponowanej}

Dla dowolnej macierzy \(A\), \((A^T)^T = A\), \((cA)^T = cA^T\), \((A + B)^T = A^T + B^T\), gdzie \(c\) to stała.

\subsection*{Wzór 34: Równanie Laplace'a}

Dla funkcji \(u(x, y)\) na płaszczyźnie, równanie Laplace'a to:
\[ \frac{\partial^2u}{\partial x^2} + \frac{\partial^2u}{\partial y^2} = 0 \]

\subsection*{Wzór 35: Równanie falowe w jednym wymiarze}

Dla fali rozchodzącej się w jednym wymiarze o prędkości \(v\), częstości \(f\) i długości fali \(\lambda\), równanie falowe to:
\[ \frac{\partial^2u}{\partial t^2} = v^2 \frac{\partial^2u}{\partial x^2} \]

\subsection*{Wzór 36: Własności liczby \(e\)}

Liczba \(e\) jest definiowana jako granica \((1 + \frac{1}{n})^n\) dla \(n \to \infty\). Ponadto, \(e^x\) to funkcja, której pochodna jest równa samej sobie: \(\frac{d}{dx} e^x = e^x\).

\subsection*{Wzór 37: Własności iloczynu wektorowego}

Dla wektorów \(\mathbf{u} = (u_1, u_2, u_3)\) i \(\mathbf{v} = (v_1, v_2, v_3)\), iloczyn wektorowy \(\mathbf{u} \times \mathbf{v}\) to wektor:
\[ \mathbf{u} \times \mathbf{v} = \begin{bmatrix} u_2v_3 - u_3v_2 \\ u_3v_1 - u_1v_3 \\ u_1v_2 - u_2v_1 \end{bmatrix} \]

\subsection*{Wzór 38: Własności całki krzywoliniowej}

Długość krzywej \(C\) opisanej przez funkcję wektorową \(\mathbf{r}(t) = (x(t), y(t), z(t))\) na przedziale \([a, b]\) to:
\[ L = \int_{a}^{b} \sqrt{\left(\frac{dx}{dt}\right)^2 + \left(\frac{dy}{dt}\right)^2 + \left(\frac{dz}{dt}\right)^2} \,dt \]

\subsection*{Wzór 39: Wzór Carnota na moc bierzącą prądu przemiennego}

Dla obwodu z prądem przemiennym o natężeniu \(I\) i napięciu \(V\) w fazie, moc bierząca to:
\[ P = IV\cos\phi \]
gdzie \(\phi\) to kąt między napięciem a prądem.

\subsection*{Wzór 40: Własności funkcji \(\Gamma\) (funkcji gamma)}

Dla dowolnego \(n \in \mathbb{N}\), funkcja \(\Gamma(n) = (n-1)!\).

\end{document}
