\documentclass{article}
\usepackage{amsmath, amssymb}

\begin{document}

\section*{Kolejne 10 Różnych Wzorów Matematycznych}

\subsection*{Wzór 51: Tożsamość trygonometryczna tangensa sumy kątów}

Dla dowolnych kątów \(\alpha\) i \(\beta\), tożsamość trygonometryczna tangensa sumy kątów to:
\[ \tan(\alpha + \beta) = \frac{\tan\alpha + \tan\beta}{1 - \tan\alpha \tan\beta} \]

\subsection*{Wzór 52: Równanie dyfuzji}

Dla funkcji gęstości \(u(x, t)\) opisującej dyfuzję, równanie dyfuzji w jednym wymiarze to:
\[ \frac{\partial u}{\partial t} = D \frac{\partial^2 u}{\partial x^2} \]

\subsection*{Wzór 53: Własności pierwiastków kwadratowych}

Dla dowolnej liczby rzeczywistej \(x\), pierwiastki kwadratowe liczby \(x^2\) to:
\[ \sqrt{x^2} = |x| \]

\subsection*{Wzór 54: Suma szeregów potęgowych}

Suma nieskończonego szeregu potęgowego \(a_0 + a_1x + a_2x^2 + \ldots\) na przedziale zbieżności to:
\[ S(x) = a_0 + a_1x + a_2x^2 + \ldots \]

\subsection*{Wzór 55: Wartość bezwzględna iloczynu}

Dla dowolnych liczb rzeczywistych \(a\) i \(b\), wartość bezwzględna iloczynu to:
\[ |ab| = |a| \cdot |b| \]

\subsection*{Wzór 56: Własności macierzy odwrotnej}

Dla macierzy odwracalnej \(A\), \(A^{-1}\) jest macierzą odwrotną, czyli \(AA^{-1} = A^{-1}A = I\), gdzie \(I\) to macierz jednostkowa.

\subsection*{Wzór 57: Wzór na obwód koła}

Dla koła o promieniu \(r\), obwód koła to:
\[ C = 2\pi r \]

\subsection*{Wzór 58: Równanie adwekcji-dyfuzji}

Dla funkcji gęstości \(u(x, t)\) opisującej adwekcję-dyfuzję, równanie adwekcji-dyfuzji w jednym wymiarze to:
\[ \frac{\partial u}{\partial t} + v \frac{\partial u}{\partial x} = D \frac{\partial^2 u}{\partial x^2} \]

\subsection*{Wzór 59: Nierówność Hoeffdinga}

Dla niezależnych zmiennych losowych \(X_1, X_2, \ldots, X_n\) spełniających warunek \(a_i \leq X_i \leq b_i\), nierówność Hoeffdinga mówi, że:
\[ P\left(\sum_{i=1}^{n} X_i - \mathbb{E}\left[\sum_{i=1}^{n} X_i\right] \geq t\right) \leq e^{-\frac{2t^2}{\sum_{i=1}^{n} (b_i - a_i)^2}} \]

\subsection*{Wzór 60: Równanie Laplace'a w sferze}

Dla funkcji \(u(r, \theta, \phi)\) opisującej pole potencjału w sferze, równanie Laplace'a w sferze to:
\[ \nabla^2 u = \frac{1}{r^2}\frac{\partial}{\partial r}\left(r^2\frac{\partial u}{\partial r}\right) + \frac{1}{r^2\sin\theta}\frac{\partial}{\partial \theta}\left(\sin\theta\frac{\partial u}{\partial \theta}\right) + \frac{1}{r^2\sin^2\theta}\frac{\partial^2 u}{\partial \phi^2} = 0 \]

\end{document}
