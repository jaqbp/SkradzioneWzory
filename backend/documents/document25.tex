\documentclass{article}
\begin{document}
Wzór 139:
\[ \frac{d}{dx}\left(\cot(x)\right) = -\csc^2(x) \]

Wzór 140:
\[ \frac{\partial^2u}{\partial t^2} = c^2 \frac{\partial^2u}{\partial x^2} \]

Wzór 141:
\[ \lim_{{n \to \infty}} \sqrt[n]{n} = 1 \]

Wzór 142:
\[ \frac{d}{dx}\left(e^{cx}\right) = c e^{cx} \]

Wzór 143:
\[ \iint_R f(x,y) \,dA = \int_{a}^{b} \int_{c}^{d} f(x,y) \,dy\,dx \]

Wzór 144:
\[ \lim_{{x \to 0}} \frac{\cos(x) - 1}{x} = 0 \]

Wzór 145:
\[ \binom{n}{k} = \binom{n-1}{k-1} + \binom{n-1}{k} \]

Wzór 146:
\[ \int_{-\infty}^{\infty} e^{-x^2} \,dx = \sqrt{\pi} \]

Wzór 147:
\[ \sum_{k=1}^{\infty} \frac{1}{k^2} = \frac{\pi^2}{6} \]

Dowód 3: \\
Rozważmy ciąg geometryczny o pierwszym wyrazie \( a \) i ilorazie \( r \). Suma \( S \) tego ciągu to
\[ S = \frac{a(1-r^n)}{1-r} \].
Gdy \( |r| < 1 \) i \( n \to \infty \), to \( S = \frac{a}{1-r} \).

\end{document}
