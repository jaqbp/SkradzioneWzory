\documentclass{article}
\begin{document}
Wzór 130:
\[ \oint_C \mathbf{F} \cdot d\mathbf{r} = \int_{a}^{b} \mathbf{F}(\mathbf{r}(t)) \cdot \mathbf{r}'(t) \,dt \]

Wzór 131:
\[ \lim_{{n \to \infty}} \frac{n!}{n^n} = 0 \]

Wzór 132:
\[ \frac{d}{dx}\left(\arcsin(x)\right) = \frac{1}{\sqrt{1-x^2}} \]

Wzór 133:
\[ \iint_S \mathbf{F} \cdot d\mathbf{S} = \iiint_V \nabla \cdot \mathbf{F} \,dV \]

Wzór 134:
\[ \lim_{{x \to 0}} \frac{\tan(x)}{x} = 1 \]

Wzór 135:
\[ \binom{n}{0} + \binom{n}{1} + \binom{n}{2} + \ldots + \binom{n}{n} = 2^n \]

Wzór 136:
\[ \int_{0}^{1} x^n \,dx = \frac{1}{n+1} \]

Wzór 137:
\[ \lim_{{n \to \infty}} \frac{1}{n} = 0 \]

Wzór 138:
\[ \sum_{n=0}^{\infty} \frac{1}{2^n} = 2 \]

Dowód 2: \\
Rozważmy równanie różniczkowe \( \frac{dy}{dx} = f(x) \). Jego ogólne rozwiązanie to
\[ y = \int f(x) \,dx + C \],
gdzie \( C \) jest stałą całkową.

\end{document}
