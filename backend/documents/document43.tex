\documentclass{article}
\usepackage{amsmath, amssymb, mathtools}

\begin{document}

\section*{Kolejne 10 Skomplikowanych Wzorów Matematycznych}

\subsection*{Wzór 71: Równanie Schrödingera dla funkcji falowej}

Dla funkcji falowej \(\psi(x, t)\) opisującej stan kwantowy, równanie Schrödingera to:
\[ i\hbar\frac{\partial \psi}{\partial t} = -\frac{\hbar^2}{2m}\frac{\partial^2 \psi}{\partial x^2} + V(x,t)\psi \]

\subsection*{Wzór 72: Równanie Maxwella dla pola elektromagnetycznego}

Równanie Maxwella opisujące pole elektromagnetyczne to:
\[ \nabla \cdot \mathbf{E} = \frac{\rho}{\varepsilon_0}, \quad \nabla \cdot \mathbf{B} = 0, \]
\[ \nabla \times \mathbf{E} = -\frac{\partial \mathbf{B}}{\partial t}, \quad \nabla \times \mathbf{B} = \mu_0\mathbf{J} + \mu_0\varepsilon_0\frac{\partial \mathbf{E}}{\partial t} \]

\subsection*{Wzór 73: Całka Fresnela}

Całka Fresnela to całka określona przez:
\[ S(x) = \int_{0}^{x} \sin\left(\frac{\pi}{2}t^2\right) \, dt \]

\subsection*{Wzór 74: Równanie Naviera-Stokesa dla płynów}

Równanie Naviera-Stokesa dla opisu ruchu płynów to:
\[ \rho\left(\frac{\partial \mathbf{v}}{\partial t} + \mathbf{v} \cdot \nabla \mathbf{v}\right) = -\nabla p + \mu\nabla^2 \mathbf{v} + \rho\mathbf{g} \]

\subsection*{Wzór 75: Równanie Helmholtza dla funkcji skalarnej}

Dla funkcji skalarnej \(\phi(\mathbf{r})\) równanie Helmholtza to:
\[ \nabla^2 \phi + k^2\phi = 0 \]

\subsection*{Wzór 76: Funkcja beta}

Funkcję beta można zdefiniować jako całkę:
\[ \Beta(p, q) = \int_{0}^{1} t^{p-1}(1-t)^{q-1} \, dt \]

\subsection*{Wzór 77: Własności operatora Laplace'a na funkcji wektorowej}

Dla funkcji wektorowej \(\mathbf{F} = (F_1, F_2, F_3)\), operator Laplace'a działa na \(\mathbf{F}\) według wzoru:
\[ \nabla^2 \mathbf{F} = (\nabla^2 F_1, \nabla^2 F_2, \nabla^2 F_3) \]

\subsection*{Wzór 78: Twierdzenie Bayesa}

Twierdzenie Bayesa dla dwóch zdarzeń \(A\) i \(B\) to:
\[ P(A|B) = \frac{P(B|A)P(A)}{P(B)} \]

\subsection*{Wzór 79: Twierdzenie Greena dla obszarów z otworami}

Dla obszaru \(D\) z otworami ograniczonym krzywą zamkniętą \(C\), twierdzenie Greena mówi, że:
\[ \oint_C (Pdx + Qdy) = \iint_D \left(\frac{\partial Q}{\partial x} - \frac{\partial P}{\partial y}\right) dA + \sum_{i} \oint_{C_i} (Pdx + Qdy) \]

\subsection*{Wzór 80: Równanie Fermiego-Pasta-Ulam-Tsingou}

Równanie Fermiego-Pasta-Ulam-Tsingou (FPUT) to układ dynamiczny opisujący nieliniowe fale w sieci krystalicznej:
\[ m_n \frac{d^2u_n}{dt^2} = k(u_{n+1} - 2u_n + u_{n-1}) + \alpha(u_{n+1} - u_n)^2 - \beta(u_n - u_{n-1})^3 \]

\end{document}
