\documentclass{article}
\begin{document}
Wzór 154:
\[ \lim_{{x \to 0}} \frac{\tan(x)}{x} = 1 \]

Wzór 155:
\[ \binom{n}{0} + \binom{n}{1} + \binom{n}{2} + \ldots + \binom{n}{n} = 2^n \]

Wzór 156:
\[ \int_{0}^{1} x^n \,dx = \frac{1}{n+1} \]

Dowód 6: \\
Rozważmy równanie różniczkowe \( \frac{dy}{dx} = f(x) \). Jego ogólne rozwiązanie to
\[ y = \int f(x) \,dx + C \],
gdzie \( C \) jest stałą całkową.

\end{document}
