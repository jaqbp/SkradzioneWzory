\documentclass{article}
\usepackage[utf8]{inputenc}
\usepackage{polski}
\usepackage[mathscr]{eucal}
\usepackage{amsmath}
\usepackage{amssymb}
\usepackage{MnSymbol}
\usepackage{amsfonts}




\begin{document}
\newcommand{\imp}{\Rightarrow}
\newcommand{\lub}{\vee}
\newcommand{\roz}{\setminus}
\newcommand{\zbp}{\emptyset}



\maketitle
Nad zadaniami myślałem sam.
\section*{Zadanie 1.}
Dowód zostanie poprowadzony przez sprzeczność. Załóżmy, że implikacja jest fałszywa. Dzieje się to jednak, wyłącznie w jednym przypadku: kiedy poprzednik jest prawdziwy, a następnik fałszywy. Wobec tego: $(p \wedge ( p \imp q ))$ jest prawdą, oraz $q$ jest fałszem. Koniunkcja $(p \wedge ( p \imp q ))$ jest prawdziwa wtedy i tylko wtedy gdy, wszystkie jej części są prawdziwe. Wobec tego $p$ jest prawdziwe, tak samo jak $p\imp q$. Jednak gdyby rzeczywiście, $p$ było prawdą, a $q$ fałszem, to implikacja $ p \imp q $ nie byłaby prawdziwa, co jest sprzeczne z poprzednim zdaniem. Uzyskana sprzeczność udowadnia, że $(p \wedge ( p \imp q )) \imp q $ jest tautologią. \blacksquare
\section*{Zadanie 2.}
Weźmy $x$ takie że:

\[ x \in A \roz(B\roz C)\]

Wtedy: 

\[ x \in A  \wedge \lnot ( x \in B \wedge x \notin C) \]

Korzystając z prawa de Mograna:

\[ x \in A  \wedge ( x \notin B \lub x \in C) \]

Z rozdzielności koniunkcji względem alternatywy otrzymujemy:

\[(x \in A \wedge x \notin B) \lub (x\in A \wedge x \in C)\]

Co upraszcza się do:

\[x \in A \roz B \lub x \in A \cap C\]

Lub innymi słowy:

\[x \in (A \roz B) \cup (A \cap C)\]

Zbiory $A  \roz ( B \roz C)$ i $ (A \roz B) \cup (A \cap C)$ są sobie równe, ponieważ każdy element jednego z nich, jest też elementem drugiego. \blacksquare

\section*{Zadanie 3.}

Niech \[A = \{0\}\]
\[B=\{0\}\]
\[C= \{0\}\]

Wtedy $ A \cup B = \{0\}$, więc $(A\cup B)\roz C = \zbp$

Z drugiej strony $A \roz C = \zbp$, więc  $(A \roz C)\cap B = \zbp$

Istotnie, dla wyżej podanych zbiorów A,B,C, równość \[ (A\cup B)\roz C = (A\roz C)\cap B\] 

jest prawdziwa
\end{document}
